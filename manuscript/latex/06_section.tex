\section{Conclusions}

\paragraph{Recommendations / Take-home message}
	Give recommendations / take-home message in one or two paragraphs

\paragraph{Limitations and future directions}
	Describe limitation with specific focus on what are your planned next steps
	in this line of research.

	As this work aimed at comparing current implementations of different methods, some limitations
	to the scope of the simulation studies were imposed by the current implementation of the different
	methods.
	Blasso has not been formally developed for multi-categorical variables, which limited the study 
	to work with missing values on variables that aare either continuous in nature or usually considered
	as such in practice.
	Furthermore, the interesting inclusion of interactions dn squared terms in the imputation models was
	not explored as their inclusion in the current MI-PCA approach is not developed to the same extent as 
	in the other methods.
	
	The resampling study compared results only for a handful of analysis models making its results difficult
	to generalize.

	Some promising methods were excluded because their current implementation did not allow to meet
	the goals of this study.
	For example, BART is currently implemented to impute the covariates for a given analysis model.
	It assumes that the dependent variable is fully observed which is something that does not fit well
	with the application on social surveys where the dependent variable is as likely as any other to be 
	afflicted by missingness.
	Furthermore, it's reliance on the definition of an analsyis model before the imputation procedure can 
	be run is considered by the authors of this paper as an undesirable feature: in general, for imputation
	of social surveys there is a preference for imputation models that can be suggested independently of
	the type and formulation of the analysis model.
