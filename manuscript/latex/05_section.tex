\section{Discussion}

In what follows, the results obtained with the simulation and the resampling studies are discussed to provide
an overall picture of how the methods compare to each other.

\paragraph{IURR and DURR effective but time consuming}
	Overall, DURR and especially IURR showed great performances in all experiments according to all performance measures.
	However, using them requires a lot more imputation time than other well performing methods.
	For example, while both IURR and MI-PCA have weaknesses, overall, they showed similar performances for most parameters 
	and performance measures.
	However, MI-PCA obtained these results in a fraction of the time taken by IURR.
	Although these time measurement are specific to our set up, with six variables to be imputed, the
	relative differences are striking and bound to have a large impact in applied research.

\paragraph{MI-PCA struggle with item variances}
	In the simulation studies, MI-PCA tended to over-estimate (positive bias) the item variances while it was the only method
	with acceptable bias for the covariances.
	In other words, it seems that MI-PCA includes a lot of uncertainty regarding the imputed values, but it recovers 
	correctly the relationships between variables.

	Furthermore, the bias for the item variances that afflicted MI-PCA in the multivariate-normal set up appears to be 
	related to the strength of the latent structure: when the latent structure is absent (experiment 1) or weak (experiment 2, 
	conditions 5 to 8, factor loadings between .5 and .6) item variances are biased, especially in the high-dimensional
	conditions; 
	when the latent structure is prominent (experiment 2, conditions 1 to 4), the variances are estimated with 
	negligible bias, even in the high-dimensional conditions.


\paragraph{MI-PCA IURR trade off}
	In the simulations, the two better performing methods, IURR and MI-PCA, exhibit different weaknesses and strengths in 
	terms of bias.
	The two simulations studies, showed that, in the high dimensional condition, MI-PCA struggles with correctly 
	recovering item variances but returns very lowly biased item covariances, while IURR has exactly the opposite 
	behaviour.

	The resampling study showed that slightly more biased estimates of the parameters in model 1 and 2 are produced 
	using MI-PCA rather than IURR, while coverage rates and confidence interval widths did not substantively differ.
	This difference did not show when looking at the focal regression coefficients, where MI-PCA performed as well 
	as IURR, if not better.
	
	Overall, the simulation studies seemed to suggest that MI-PCA could recover more accurately the relationships
	between variables (less biased covariances) than IURR, while in the resampling study this conclusion was not
	supported to the same extent: when looking at bias and coverage rates for $\beta_{1,1}$ and $\beta_{1,2}$, MI-PCA 
	performed equally well or better than DURR and IURR, while looking at the overall measures of bias and coverages
	IURR and DURR outperformed MI-PCA.

\paragraph{Bridge inadequacy for high-dimensional set ups}
	In both the simulation and resampling study the use of a fixed ridge penalty within the imputation
	algorithm to facilitate the inversion of the observed data matrix lead to extreme bias in the high
	dimensional conditions.

\paragraph{MI-CART and MI-RANF}
	[UNFINISHED] A few words on these methods. Probably need to pay more attention to them in the result section as well.
