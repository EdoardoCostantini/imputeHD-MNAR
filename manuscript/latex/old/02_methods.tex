\maketitle
\section{Methods}

\maketitle
\subsection{Experiment 1: Multivaraite Normal Data}

The performance of the selected methods was compared through a Monte Carlo simulation study. Performances
were assessed in terms of how well the incomplete-data analysis remained statistically valid after missing 
data was treated. Assuming the complete-data analysis is statistically valid, MI should allow for unbiased
and confidence valid analysis of the incomplete data.

\subsubsection{Experimental Conditions}

Two design parameters were varied in the simulation: the proportion of (per variable) missing cases ($pm$), with levels $\{.1, .3\}$; 
and the number of features of the dataset ($p$), with levels $\{50, 500\}$. 
In all conditions, 200 observations were generated ($n = 200$) and the entire data set was considered when 
conducting imputations, so that the higher dimensionality of the data resulted in a higher number of potential 
auxiliary variables.

\begin{table}[h]
	\centering
	\caption{Experiment 1 conditions ($n = 200$)}
	\break
	\begin{tabular}{ c c c c }
		Cond 	& n   & pm 	& p 	\\ 
		 \hline
		 1	& 200 & .1 	& 50  	\\  
		 2	& 200 & .1 	& 500 	\\
		 3	& 200 & .3 	& 50  	\\
		 4	& 200 & .3 	& 500 	\\
		 \hline
	\end{tabular}
	\label{table:exp1_conds}
\end{table}

500 data sets were generated for each condition. After imputation, the MLE estimates and standard errors of 
the means, variances and covariances, of the variables originally with missing values, were estimated and 
pooled across multiply imputed datasets according to \citet{rubin:1987}'s rules.

\subsubsection{Data Generation}

A dataset X, with dimensionality $n \times p$ (with n = number of observations and p = number of features),
was generated according to the standard normal multivariate model:

\begin{equation}
	X = MVN(\mu_0 = \bf{0}, \Sigma_0) \label{eq:MVN}
\end{equation}

where $\mu_0$ is a $p \times 1$ vector of 0s, and $\Sigma_0$ is a $p \times p$ correlation matrix. Variables were divided in 
three correlation blocks: high, mid and low correlation. Five variables belonged to block 1 and were correlated among themselves with 
a correlation coefficient $\rho_1 = .6$. Another five variables belonged to block 2 and were correlated among themselves, and with variables in 
block 1, with $\rho_2 = .3$. All remaining variables belonged to block 3 and were correlated among themselves, and with variables in 
block 1 and 2, with $\rho_3 = .01$. 

$$
\begin{matrix}
		& Block 1 	& Block 2	& Block 3	\\
	Block 1	& \rho_1	& \rho_2 	& \rho_3 	\\
	Block 2 & \rho_2	& \rho_2	& \rho_3 	\\
	Block 3	& \rho_3 	& \rho_3	& \rho_3 	\\
\end{matrix}
$$

After sampling the values of X from equation \ref{eq:MVN}, all columns were rescaled to match means, variances, and 
covariances of continuously treated items in EVS waves. This was done to facilitate the interpretation of the findings
in terms of real data applications. 

\iffalse %comment out this section for now
Looking at 10 points EVS items on democratic beliefs and moral values in the 2017 
EVS wave, we noticed that on average these items have means and variances around 5. Hence, each variable in X was centred 
around 5 (instead of 0) and scaled to have variance around 5 (instead of 1). As a result the covariances between variables
were also rescaled to match EVS data: correlations of .6 and .3 translated to covariances of 3 and 1.5 respectively.
\fi

\subsubsection{Missing Data Imposition}

Missing values were imposed on six variables, three in block 1 and three in block 2, using the cumulative logistic distribution
to define the probability of missingness based on a linear combination of four scaled columns of $X$.

\begin{equation}
	P(y_t = MISS | X) = G(\tilde{X} \theta) \label{eq:PBT_imp}
\end{equation}

where $y_t$ is a variable target of missing values imposition ($t = 1,...,T$, with $T$ = 6), $G$ is the standard cumulative logistic distribution 
(with location and scale parameters equal to 0 and 1 respectively), $\tilde{X}$ is a $n \times 4$ standardized subset of $X$, 
including only the determinants of missingness, and $\theta$ is a vector of regression coefficients. 

$\tilde{X}$ was composed of 2 variables from block 1 and 2 from block 2. Of the two variables from both blocks,
one was selected as target of missing values itself, and the other was fully observed. All variables have same weight
in the linear combination $\tilde{X}\theta$.

\iffalse %comment out this section for now
The value of the linear predictor was also "off-setted" to induce missingness in the lower tail of the distribution
of linear predictor $\tilde{X}\theta$. This procedure does not imply that missingness in the target variable depends 
on the target variable itself, as the offset is applied based on the values of $\tilde{X}$.
\fi

Overall, the missing data imposition procedure allowed to: 
(1) specify a general missing data pattern; 
(2) work with a MAR missingness set up;
(3) impose a desired proportion of missing values on each target variable; 
(4) induce substantive bias for parameters estimates of complete case analysis (between 20 and 30 percent of 
the size of the reference "true" value of the parameter).

\subsubsection{Imputation Methods and Analyses Model}

For each simulated data set, imputation was performed according to all the methods referenced to in the introduction.

As traditional parametric MI could not be applied to cases with $p > n$, we ran, for reference, a standard mice imputation 
routine that used all variables in block 1 and 2 (10 in total), which included all predictors of missingness and no 
auxiliary junk variables. Results from this "oracle" run are referred to as MI Optimal run (MI-OP). 
We have also considered two single dataset missing data handling approaches: MissForest \citep{stekhovenBuhlmann:2011} as 
implemented in the R package 'misForest' (here referred to as 'missFor'), and Complete Case analysis (CC), performed 
using only complete rows of the data.

Convergence of the imputations was checked through visual examination of trace plots showing the mean imputed values
for each variable at each iteration. In the most complex condition, ($p = 500, pm = .3$), all methods converged after 
after approximately 20 iterations. In the simulation study we run a single chain of the MI algorithms for 50 
iterations, considering the first 20 as burn-in, and selecting 10 imputed datasets through thinning.

Maximum Likelihood Estimates of the means, variances, and covariances of the six variables with missing values  
were obtained by fitting a saturated model to the treated data, and pooling the multiple estimates when necessary. 
Finally, "Gold Standard" (GS) MLEs of the same parameters were obtained from the fully observed data sets (before missing data imposition).


\maketitle
\subsection{Experiment 2: Data with latent structure}

The performance of the selected methods was compared in the presence of a latent structure. It is expected 
that the PcAux method will perform better than the others, given how it extracts components from the data 
that might overlap with the latent constructs. Performances were assessed as in experiment 1, in terms of 
how well the incomplete-data analysis remained statistically valid after missing data was treated. 

\subsubsection{Experimental Conditions}

Three design parameters were varied in the simulation: the proportion of (per variable) missing cases ($pm$),
with levels $\{.1, .3\}$, as in the first experiment, the number of latent variables ($p$), with levels $\{10, 
100\}$, and the size of the factor loadings ($\lambda = {high, low}$). 5 items were generated in all crossed
coditions for each of the latent vairables.

In all conditions, 200 observations were generated ($n = 200$) and 5 items measuring each latent variable 
were generated so that when conducting imputations, the conditions with 100 latent variables resulted 
in a number of potential imputation model auxiliary variables higher than the number of observations
available.

\begin{table}[h]
	\centering
	\caption{Experiment 1 conditions ($n = 200$)}
	\break
	\begin{tabular}{ c c c c }
		Cond & fl & pm & lv \\ 
		\hline
		1 & high & 0.10 & 10 \\ 
		2 & high & 0.10 & 100 \\ 
		3 & high & 0.30 & 10 \\ 
		4 & high & 0.30 & 100 \\ 
		5 & low & 0.10 & 10 \\ 
		6 & low & 0.10 & 100 \\ 
		7 & low & 0.30 & 10 \\ 
		8 & low & 0.30 & 100 \\
		\hline
	\end{tabular}
	\label{table:exp1_conds}
\end{table}

500 data sets were generated for each condition. After imputation, the MLE estimates and standard errors of 
the means, variances and covariances, of the variables originally with missing values, were estimated and 
pooled across multiply imputed datasets according to \citet{rubin:1987}'s rules.

\subsubsection{Data Generation}

A dataset X, with dimensionality $n \times 5\delta$ (with n = number of observations and $\delta$ = number 
of latent variables), was generated based on the following latent variable model:

\begin{equation}
	INSERT MATH HERE
\end{equation}

with $\Phi$, $\Theta$ being the latent variable covariance matrix and items error covairance matrix.
$\lambda$ was sampled from a uniform distribution with bounds .9 and .97, in the conditions with 
high values of the factor loadings, and .5, and .6, in the conditions wiht low values of the factor loadings.
The first set of bounds was chosen on the lines of what has been done in previuos research (see Hastie Elastic 
Net paper), while the second was chosen to mimic scales administred in social survey.
In all of the factorial conditions, 4 latent vairables were highly correlated ($\rho = .6$), another
4 latent vairables were only correlated by $\rho = .3$ among themselves and with the highly correlated latent
variables. The remaining number of latent variables were independently generated around a mean score of 0.
Items were then generated assuming no correlation between their measurment error.

\subsubsection{Missing Data Imposition}

BOOKMARK!

Missing values were imposed on six variables, three in block 1 and three in block 2, using the cumulative logistic distribution
to define the probability of missingness based on a linear combination of four scaled columns of $X$.

\begin{equation}
	P(y_t = MISS | X) = G(\tilde{X} \theta) \label{eq:PBT_imp}
\end{equation}

where $y_t$ is a variable target of missing values imposition ($t = 1,...,T$, with $T$ = 6), $G$ is the standard cumulative logistic distribution 
(with location and scale parameters equal to 0 and 1 respectively), $\tilde{X}$ is a $n \times 4$ standardized subset of $X$, 
including only the determinants of missingness, and $\theta$ is a vector of regression coefficients. 

$\tilde{X}$ was composed of 2 variables from block 1 and 2 from block 2. Of the two variables from both blocks,
one was selected as target of missing values itself, and the other was fully observed. All variables have same weight
in the linear combination $\tilde{X}\theta$.

\iffalse %comment out this section for now
The value of the linear predictor was also "off-setted" to induce missingness in the lower tail of the distribution
of linear predictor $\tilde{X}\theta$. This procedure does not imply that missingness in the target variable depends 
on the target variable itself, as the offset is applied based on the values of $\tilde{X}$.
\fi

Overall, the missing data imposition procedure allowed to: 
(1) specify a general missing data pattern; 
(2) work with a MAR missingness set up;
(3) impose a desired proportion of missing values on each target variable; 
(4) induce substantive bias for parameters estimates of complete case analysis (between 20 and 30 percent of 
the size of the reference "true" value of the parameter).

\subsubsection{Imputation Methods and Analyses Model}

For each simulated data set, imputation was performed according to all the methods referenced to in the introduction.

As traditional parametric MI could not be applied to cases with $p > n$, we ran, for reference, a standard mice imputation 
routine that used all variables in block 1 and 2 (10 in total), which included all predictors of missingness and no 
auxiliary junk variables. Results from this "oracle" run are referred to as MI Optimal run (MI-OP). 
We have also considered two single dataset missing data handling approaches: MissForest \citep{stekhovenBuhlmann:2011} as 
implemented in the R package 'misForest' (here referred to as 'missFor'), and Complete Case analysis (CC), performed 
using only complete rows of the data.

Convergence of the imputations was checked through visual examination of trace plots showing the mean imputed values
for each variable at each iteration. In the most complex condition, ($p = 500, pm = .3$), all methods converged after 
after approximately 20 iterations. In the simulation study we run a single chain of the MI algorithms for 50 
iterations, considering the first 20 as burn-in, and selecting 10 imputed datasets through thinning.

Maximum Likelihood Estimates of the means, variances, and covariances of the six variables with missing values  
were obtained by fitting a saturated model to the treated data, and pooling the multiple estimates when necessary. 
Finally, "Gold Standard" (GS) MLEs of the same parameters were obtained from the fully observed data sets (before missing data imposition).


\maketitle
\subsection{Experiment 4: Resampling Study}

\subsubsection{Analysis models}

In this study we defined two analysis models following two sociological published articles: 
Immerzeel Et Al 2017 and Koneke 2014. As a result we considered:
\begin{itemize}
	\item a linear regression of euthanastia attitutes on institutional trust measures (general trust, 
		trust in the state, etc.) and other covariates
	\item a linear regression of left/right voting tendencies on sex, SES, religion and other covariates
\end{itemize}

\paragraph{Model 1: Euthanasia acceptance}
For now, many vairbales have been recoded, I actually do not thing I will keep this.
(results are ofcourse the same just with inverted signs of the variables with recoded
order; for imputation this is entierly irrelevant, for analysis models might be
a comment but not so relevant)

\begin{itemize}
	\item euthanasia acceptance (v156) (DV)
	\item General Trust (v31) Dichotomous variable (1 = having general trust)
	\item Confidence in healtcare sustem (v126) (1 to 4, 1 = low, 4 = high)
	\item Confidence in press (v118) (1 to 4, 1 = low, 4 = high)
	\item Confidence in state (v120, v121, v127. v131), items recoded  (1 to 4, 
		1 = low, 4 = high), mean taken.
	\item Age (age)
	\item Education (v243\_ISCED\_1) treated as continuous
	\item Sex (v225, male = 1)
	\item Religiousness (v6) recoded 0 to 3, with 0 not religious, 3 high
	\item Denomination (v51v52), kept all categories in dataset
\end{itemize}

\paragraph{Model 2: Left/Right voting tendencies}

Variables were oberationalized as follow:
\begin{itemize}
	\item LEFT\/RIGHT vote (v174\_LR): a continuous vairables in the range from 1 (left) to 10 (right)
		(this is DV, see papers for description)
	\item sex (female = 1)
	\item nativist attitudes, as made up of items v185 to v187, and interval mean scale
	\item low and order: opinion on strong leader (v145); opinion on order (v110)
	\item political interest (v97)
	\item political action as interval mean scale based on items v98:v1010
	\item SES (v246\_egp) treated as categorical
	\item Age (age) 
	\item Education (v243\_ISCED\_1)
	\item Marital Status based on v234 recoded into 3 categories (yes, no, never)
	\item Religiousness (v54) as frequence of attendence recoded to 3 cateogries of frequency (0 = none, 2 = high)
	\item Denomination (v51v52), kept all categories in dataset
	\item Urbanization of area based on v276\_r		
	\item Religion: in the reference paper the question is a combination of do you belong to religion? if yes, which?
		In EVS, this is question v51 and v52. Maybe there is a version that combines the info. Check.
	\item Degree of religiosity: is there an EVS question for this? I think so, check (0-10). v54 to v56 cover this a bit.
	\item Gender: Need to find in EVS v225
	\item Parents Origin: how? Both parents separately? Need to combine father born in same country v230, yes/no; and 
		v231b country of birth of father. Same for mother (v232 and v232b)
\end{itemize}

\subsubsection{Variable Operationalization}

\subsubsection{Response model}
Missingness is imposed accoridng to the usual algorithm that generates a response vector
with a desired proportion of missing values based on a linear combination of 1 or more
predictors (with weighted importance). As predictors we have for now selected education (v243\_ISCED\_1),
and trust (v34) in people the interviewee does not know. The simplistic non-response mechanisms 
is that people with low eductaion and low trust tend to avoid answering more questinos.

Six vairables are identified as target of missings: the 3 vairables making up the interval scale
for nativist attitudes (IV scale), left/right vote (DV), trust in the press (IV item), euthanesia 
acceptance (DV).




\subsection{Evaluation Criteria}

Estimation bias introduced by the missing data treatment was quantified as Percent Relative Bias (PBR):

\begin{equation}
	PBR = \frac{\bar{Q}^{k} - R^{k} }{R^{k}}*100 \label{eq:bias_p}
\end{equation}

where $\bar{Q}^{k}$ is the mean estimate of parameter $k$ across the Monte Carlo simulations, and $R^{k}$ is the 
reference value corresponding to that parameter. The reference ("true") values of the parameters of interest were 
obtained by averaging the 500 MLEs obtained on the fully observed datasets. 

Furthermore, the euclidean distance $d$ between vectors of raw parameter estimates of the same type of statistic 
($\bm{Q}^{K}$) and a reference $\bm{R}^{K}$ vector was considered to provide a more aggregate quantification of bias:

\begin{equation}
	d(\bm{R}^{K}, \: \bm{Q}^{K}) = 
		\sqrt{ 
			(R^{K}_{1} - Q^{K}_{1})^{2} + 
			(R^{K}_{2} - Q^{K}_{2})^{2} + 
			... + 
			(R^{K}_{T} - Q^{K}_{T})^{2}
		} 
			\label{eq:eu_dist}
\end{equation}

where $\bm{Q}^{K}$ and $\bm{R}^{K}$ are vectors of parameters estimates of statistic type $K$ (i.e., means, variances, covariances). 
In particular, $\bm{R}^{K}$ is the vector or reference values, and $\bm{Q}^{K}$ is a vector of Monte Carlo parameters estimates after 
missing data treatment, and $T$ is the number of variables with missing values.

Finally, to assess the integrity of hypothesis tests conducted under the various imputation approches, the 95\% confidence 
interval coverage rates were computed as:

\begin{equation}
	CI_{cov} = \frac{ \sum_{s=1}^{S} I(Q \in \hat{CI}_{s} ) }{S^{k}}*100 \label{eq:ci_cov}
\end{equation}

\iffalse %comment out this section for now

\subsection{Experiment 2: Interactions and the like}
Data for this experiment were generated in two steps. First, a matrix of predictors was generated as in experiment 1.
Then, a dependent variable y was generated using one predictor from each block. Depending on the condition, an 
interaction term was either included or not.

Missing data was imposed using a probit model to facilitate manipulation of the proportion of missing cases.
In all conditions, seven variables where targeted by missingness: y, three in block 1, and three in block 2. Four variables
where randomly selected from X to use as predictors in the probit model for each target variable. The same set of
coefficients where used ($-1, -1, .666, -.333$).

\subsection{Experiment 3: Latent data}
Bla bla this makes a lot of sense

\fi
