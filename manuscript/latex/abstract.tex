% Write you abstract here

Including many predictors in the imputation models specified for a Multiple Imputation (MI) 
procedure is one of the most challenging tasks imputers face.
A variety of high-dimensional data MI techniques can facilitate the task, but there has been 
limited research on their relative performances.
In this study, we investigate a wide range of extant high-dimensional data MI (HD-MI) techniques 
that can handle a large number of predictors in the imputation models and general missing data 
patterns.
The relative performance of seven HD-MI methods is assessed with extensive numerical studies.
The quality of imputations is defined by the degree to which they grant unbiased and confidence 
valid estimates of the parameters of analysis models.
We found that using regularized regression within MI chained equation algorithm to select active 
predictors, and using Principal Components to reduce the dimensionality of auxiliary data were 
the strongest performers.
