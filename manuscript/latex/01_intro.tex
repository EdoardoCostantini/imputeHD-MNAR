\maketitle
\section{Introduction}

Today’s social and behavioral scientists are blessed with a wealth of large, high-quality and publicly available social scientific datasets such as the Longitudinal Internet Studies for the Social Sciences (LISS) Panel and the European Values Study (EVS), with initiatives being undertaken to link and extend these datasets into a full system of linked open data (LOD). Making use of the full potential of these data sets requires dealing with the crucial problem of missing data. 

The tools researchers working with these data sets need to correct for the bias introduced by nonresponses 
require special attention. The large number of items recorded, coupled with the longitudinal nature of surveys and the necessity
of preserving complex interactions and nonlinear relations, easily produces high-dimensional ($p>n$) imputation 
problems that impair a straightforward application of imputation algorithms such as MICE \citep{vanBuuren:2012}.

Furthermore, when employing Multiple Imputation to deal with missing values, data handlers tend to prefer including more
predictors in the imputation models as to reduce chances of uncongenial imputation and analysis models \citep{meng:1994}.
High-dimensional data imputation settings represent both an obstacle and an opportunity in this sense: an 
obstacle, as in the presence of high-dimensional data it is simply not possible to include all variables in standard parametric
imputation models; an opportunity, because the large amount of features available has the potential to reduces the chances of 
leaving out of the imputation models important predictors of missignenss.

Many solutions have been proposed to deal with missing values in high dimensional contexts, but most of them
have focused on single imputations in an effort to improve the accuracy of individual imputations \citep{kimEtAl:2005, 
stekhovenBuhlmann:2011, d'ambrosioEtAl:2012}.
The main task of social scientists is to make inference about a general population based on a sample of observed 
data, and single imputation is simply an inadequate missing data handling technique for such purpose: it 
does not guarantee to find estimates that are unbiased and confidence valid \citep{rubin:1996}.

Recent years have seen a plethora of studies proposing new Multiple Imputation methods for high dimensional
missing data, which can be grouped as follows: 
\begin{itemize}
\item MI through frequentist use of regularized regression - Using regularized regression to
	deal with the high-dimensionality of imputation models has been proposed and tested by
	\citet{zhaoLong:2016} and \citet{dengEtAl:2016}. Their methods are referred to here 
	as Direct and Indirect Use of Regularized Regression (DURR, and IURR, as abbreviated in their papers);
\item MI through Bayesian regularized regression - In the R package 'mice', the implementation of 
	Bayesian MI under the normal linear model allows to use a ridge penalty to estimate the imputation 
	model regression coefficients in the presence of high collinearity and more features than observations
	\citep[p. 68, algotithm 3.1]{vanBuuren:2012}. In the present work, we refer to such approach as 'bridge'. 
	\citet{zhaoLong:2016} have also proposed using \citet{hans:2009}'s Bayesian Lasso regression to implement an 
	alternative fully Bayesian high-dimensional imputation approach (blasso).
\item non-parametric MI through regression trees - To this category belong the methods of 
	\citet{burgetteReiter:2010, shahEtAl:2014} who proposed, respectively, an integration of regression trees 
	and random forest within the Multiple Imputation by Chained Equations algorithm. In this paper they are 
	referred to as MI-CART and MI-RANF.
\item MI through dimensionality reduction - \citet{howardEtAl:2015} proposed using PCA to extract a few components
	from a set of auxiliary variables to be used as predictors in regular runs of a MICE algorithm bringing the 
	imputation problem back to a low dimensional one. Such method is referred to here as MI-PCA.
\end{itemize}
Despite being so prolific, the field is in need of comprehensive studies that compare the performances
of such state-of-the-art methods on fair ground for specific scientific endeavours.

With this study we set out to present a thorough review and comparison of MI approaches for high-dimensional 
datasets. The goal was assessing how they meet the requirements of statistical validity of the analysis 
performed on the treated data.

The following article is organized as follows (40 pages manuscript styled pages):
\begin{itemize}
 \item Introduction (done)
	Frame problem. Discuss background literature.
	Main focus: what is the reason to write paper
	With these sections coming up.

 \item Algorithms and Imputation methods
	- DURR
	- IURR
	- blasso
	etc
	- last one
	Focus on minimal possible description to give reader sense of what the method is
	without going to the source paper. (deng et al max). Do not feel onbligated to use
	eq, prefer words: use it if it adds specificity and clearity.
	Optuon can be having extenisve definition in appendix and very strimlined 
	here. In appendix defineltyy all teh "adaptation" that were required.
 \item Simulation Studies

  \begin{itemize}
   \item Methods
	Study 1 + Study 2
   	    \item Data geneartion
	    \item Missing data imposition
	    \item Analysis models
	    \item Criteria
		\item procedure: summary of crossed conditions: describe sequentially 
			what happens each replication
   \item Results
  \end{itemize}

 \item Resampling Study (EVS)
  \begin{itemize}
	\item Methods
	   \item Data preparation (givign currect documentation for the data, what it is, 
		   why collectedcleaning, with general demographics originally + went to
		   systematic cleaning process with general purpuse for the cleaning + 
		   large western european countries, details are in the appenxid if 
		   interested)
	   \item Missing data imposition
	   \item Analysis models
	   \item Criteria
	   \item Procedure (summary with the number of observations kept and so on)
	\item Results/Discussion
		again divide by type. with some implication but not comparison
  \end{itemize}

 \item Discussion
	Syntetize findings: here make parallels and comparisons.

 \item Conclusions 
	 One or two paragraphs with take home, limitations, future directions (hint at 
	 MY future work)

 \item Appendices
 - methods detials
 - EVS quirks
\end{itemize}


