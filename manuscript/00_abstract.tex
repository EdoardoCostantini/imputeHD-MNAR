% Write you abstract herehttps://www.overleaf.com/project/604268896fb0ac4a7209eb37

Including a large number of predictors in the imputation model underlying a Multiple Imputation (MI) 
procedure is one of the most challenging tasks imputers face.
A variety of high-dimensional MI techniques (HD-MI) can facilitate this task, but there has been 
limited research on their relative performance.
In this study, we investigate a wide range of extant HD-MI techniques 
that can handle both large numbers of predictors in the imputation model and general missing data patterns.
We assess the relative performance of seven HD-MI methods with two Monte Carlo simulation studies and a 
resampling study based on real survey data.
The performance of the methods is defined by the degree to which they facilitate unbiased and confidence-valid estimates of the parameters of complete data analysis models.
We find that using regularized regression to select the predictors used in the MI model, and using principal components analysis to reduce the dimensionality of auxiliary data produce the best results.